% !TeX root = RJwrapper.tex
\title{Blocking: An R Package for Blocking of Records for Record Linkage and Deduplication}


\author{by Maciej Beręsewicz and Adam Struzik}

\maketitle

\abstract{%
An abstract of less than 250 words.
}

\section{Introduction}\label{introduction}

Interactive data graphics provides plots that allow users to interact them. One of the most basic types of interaction is through tooltips, where users are provided additional information about elements in the plot by moving the cursor over the plot.

This paper will first review some R packages on interactive graphics and their tooltip implementations. A new package \CRANpkg{ToOoOlTiPs} that provides customized tooltips for plot, is introduced. Some example plots will then be given to showcase how these tooltips help users to better read the graphics.

\section{Background}\label{background}

Some packages on interactive graphics include \CRANpkg{plotly} \citep{plotly} that interfaces with Javascript for web-based interactive graphics, \CRANpkg{crosstalk} \citep{crosstalk} that specializes cross-linking elements across individual graphics. The recent R Journal paper \CRANpkg{tsibbletalk} \citep{RJ-2021-050} provides a good example of including interactive graphics into an article for the journal. It has both a set of linked plots, and also an animated gif example, illustrating linking between time series plots and feature summaries.

\section{\texorpdfstring{Blocking of records using \texttt{blocking} function}{Blocking of records using blocking function}}\label{blocking-of-records-using-blocking-function}

\section{Integration with existing packages}\label{integration-with-existing-packages}

\section{Case study}\label{case-study}

\section{\texorpdfstring{Customizing tooltip design with \pkg{ToOoOlTiPs}}{Customizing tooltip design with }}\label{customizing-tooltip-design-with}

\pkg{ToOoOlTiPs} is a packages for customizing tooltips in interactive graphics, it features these possibilities.

\section{A gallery of tooltips examples}\label{a-gallery-of-tooltips-examples}

The \CRANpkg{palmerpenguins} data \citep{palmerpenguins} features three penguin species which has a lovely illustration by Alison Horst in Figure \ref{fig:penguins-alison}.

\begin{figure}
\includegraphics[width=1\linewidth,height=0.3\textheight,alt={A picture of three different penguins with their species: Chinstrap, Gentoo, and Adelie. }]{figures/penguins} \caption{Artwork by \@allison\_horst}\label{fig:penguins-alison}
\end{figure}

Table \ref{tab:penguins-tab-static} prints at the first few rows of the \texttt{penguins} data:

\begin{table}
\centering
\caption{\label{tab:penguins-tab-static}A basic table}
\centering
\fontsize{7}{9}\selectfont
\begin{tabular}[t]{l|l|r|r|r|r|l|r}
\hline
species & island & bill\_length\_mm & bill\_depth\_mm & flipper\_length\_mm & body\_mass\_g & sex & year\\
\hline
Adelie & Torgersen & 39.1 & 18.7 & 181 & 3750 & male & 2007\\
\hline
Adelie & Torgersen & 39.5 & 17.4 & 186 & 3800 & female & 2007\\
\hline
Adelie & Torgersen & 40.3 & 18.0 & 195 & 3250 & female & 2007\\
\hline
Adelie & Torgersen & NA & NA & NA & NA & NA & 2007\\
\hline
Adelie & Torgersen & 36.7 & 19.3 & 193 & 3450 & female & 2007\\
\hline
Adelie & Torgersen & 39.3 & 20.6 & 190 & 3650 & male & 2007\\
\hline
\end{tabular}
\end{table}

Figure \ref{fig:penguins-ggplot} shows an plot of the penguins data, made using the \CRANpkg{ggplot2} package.

\begin{verbatim}
penguins %>% 
  ggplot(aes(x = bill_depth_mm, y = bill_length_mm, 
             color = species)) + 
  geom_point()
\end{verbatim}

\begin{figure}
\includegraphics[width=1\linewidth]{paper-blocking_files/figure-latex/penguins-ggplot-1} \caption{A basic non-interactive plot made with the ggplot2 package on palmer penguin data. Three species of penguins are plotted with bill depth on the x-axis and bill length on the y-axis. Visit the online article to access the interactive version made with the plotly package.}\label{fig:penguins-ggplot}
\end{figure}

\section{Summary}\label{summary}

We have displayed various tooltips that are available in the package \pkg{ToOoOlTiPs}.

\section{Acknowledgements}\label{acknowledgements}

Work on this package is supported by the National Science Centre, OPUS 20 grant no. 2020/39/B/HS4/00941

\bibliography{RJreferences.bib}

\address{%
Maciej Beręsewicz\\
University of Economics and BusinessStatisical Office in Poznań\\%
Department of Statistics, Poznań, Poland\\ Centre for the Methodology of Population Studies\\
%
\url{https://maciejberesewicz.com}\\%
\textit{ORCiD: \href{https://orcid.org/0000-0002-8281-4301}{0000-0002-8281-4301}}\\%
\href{mailto:maciej.beresewicz@poznan.pl}{\nolinkurl{maciej.beresewicz@poznan.pl}}%
}

\address{%
Adam Struzik\\
Adam Mickiewicz UniversityStatisical Office in Poznań\\%
Department of Mathematics, Poznań, Poland\\ Centre for Urban Statistics\\
%
%
%
\href{mailto:adastr5@st.amu.edu.pl}{\nolinkurl{adastr5@st.amu.edu.pl}}%
}
