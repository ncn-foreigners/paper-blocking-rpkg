% !TeX root = RJwrapper.tex
\title{blocking: An R Package for Blocking of Records for Record Linkage and Deduplication}


\author{by Maciej Beręsewicz and Adam Struzik}

\maketitle

\abstract{%
Entity resolution (probabilistic record linkage, deduplication) is essential for bringing together data needed for estimation or other analyses based on multiple sources. It aims to link records that refer to the same entity (e.g., person, company) in absence of unique identifiers. Without identifiers, researchers must specify which records to compare to calculate matching probability and reduce computational complexity. Traditional deterministic blocking uses common variables like first letters of names, sex, or dates of birth, but assumes error-free, complete data. To address this limitation, we developed the R package \textbf{blocking}, which uses approximate nearest neighbor search and graph algorithms to reduce the number of comparisons. This paper presents the package design, functionalities, and two case studies.
}

\section{Introduction}\label{introduction}

\subsection{Blocking for record linkage}\label{blocking-for-record-linkage}

Entity resolution (probabilistic record linkage, deduplication) is
essential for bringing together data needed for estimation or other
analyses based on multiple sources
(cf.~\citet{fellegi1969theory}, \citet{Binette2022}). The goal is to link records that refer to the same entity (e.g., person,
company, job position) in the absence of unique identifiers. This situation is frequently observed in
administrative records, e.g., for foreign-national populations. For
instance, the Social Insurance Institution register in Poland at the end
of 2023 included 1.206 million records referring to approximately 1.105
million individuals, of which about 10\% had missing information in the
personal identifier (PESEL) and about 50\% had missing address details.
Note that the exact number of individuals is certainly lower than 1.105
million, as the 10\% with missing identifiers may include duplicates
(cf.~\citet{beresewicz2025estimation}).

This drives the need to link records without unique identifiers, which often
requires certain assumptions about how to reduce the large number of
possible comparisons, as it is not feasible to compare all pairs of
records in large datasets (e.g., the aforementioned example would
require over 700 billion comparisons). Consequently, researchers aim to
reduce the number of comparisons in various ways prior to the record
linkage/deduplication stage. The rationale is twofold: computational
resource constraints and clerical review workload.

Reducing the number of comparisons is accomplished through blocking, a
method that limits possible comparisons by assuming that certain
variables must match exactly or that combinations of variables should
match above a specified threshold. For instance, a standard approach
assumes that sex or birth year must match exactly, whilst other record
characteristics may vary. Another method employs phonetic algorithms
such as Soundex (cf.~\citet{Wright1960}) or its adaptations for non-English
languages (cf. \citet{Howard2020}) to block records that sound similar but
are spelt differently (e.g., Smith and Smyth, or Anna and Ania).
Furthermore, with the growing popularity of large language models (both
closed and open-source), one may consider using embeddings
\citep{mikolov2013efficient} to identify nearest neighbours and treat these
as potential comparison pairs. For a comprehensive review of blocking
methods, see \citet{Steorts2014} or \citet{Papadakis2020}. Section~1.2 discusses
existing R packages that implement blocking methods.

Reducing the number of pairs has inherent costs: missed comparisons lead
to increased false positive rates (FPR) and false negative rates (FNR)
in linkage studies. To validate the blocking strategy, a subset of true pairs
should be provided, or simulation studies of proposed methods should be
conducted. Alternatively, one may consider approaches proposed by
\citet{dasylva2021estimating} and \citet{dasylva2022consistent}, who demonstrated how
to estimate FPR and FNR without access to an audit sample.

\subsection{Existing software and our contribution}\label{sec-software}

The R ecosystem offers several packages that implement various blocking
techniques, which we have grouped according to the following
classification:

\begin{itemize}
\tightlist
\item
  \textbf{Deterministic blocking}:

  \begin{itemize}
  \tightlist
  \item
    \CRANpkg{reclin2} (\citet{reclin2}, \citet{reclin2-rjournal}) allows pairing
    records using \texttt{pair\_blocking()} with a prespecified list of
    columns in a \texttt{data.frame}, and the \texttt{pair\_minsim()} function,
    which allows specifying the minimal similarity score (e.g., 1
    out of 3 variable values must match exactly).
  \item
    \CRANpkg{RecordLinkage} (\citet{RecordLinkage}, \citet{RecordLinkage-rjournal})
    allows specifying blocking variables in the \texttt{blockfld} parameter
    of either \texttt{compare.dedup()} or \texttt{compare.linkage()} functions as
    a vector (either character or numeric).
  \item
    \CRANpkg{fastLink} (\citet{fastLink}, \citet{enamorado2019using}) implements
    various blocking methods via the \texttt{blockData()} function,
    including exact matching, window matching (e.g., no more than
    2-year difference between birth years), and \(k\)-means clustering.
    Notably, \texttt{fastLink} returns datasets split into separate lists,
    whilst \texttt{reclin2} and \texttt{RecordLinkage} packages create a single
    dataset.
  \end{itemize}
\item
  \textbf{Phonetic blocking}:

  \begin{itemize}
  \tightlist
  \item
    \CRANpkg{RecordLinkage} allows direct specification of phonetic
    comparison via the \texttt{phonetic} argument in \texttt{compare.dedup()} or
    \texttt{compare.linkage()} functions using the \texttt{soundex()} function.
  \item
    Additionally, \CRANpkg{stringdist} \citep{stringdist} implements the
    Soundex algorithm, whilst \CRANpkg{phonics} \citep{phonics} implements various phonetic algorithms that can be applied prior to the blocking procedure (e.g., to create a
    new column).
  \end{itemize}
\item
  \textbf{Probabilistic blocking}:

  \begin{itemize}
  \tightlist
  \item
    \CRANpkg{klsh} \citep{klsh} is the only R package that implements
    probabilistic blocking using the \(k\)-means variant of locality
    sensitive hashing. The main \texttt{klsh()} function implements this
    approach, and the resulting object is a list containing row
    identifiers for the prespecified number of blocks (via the
    \texttt{num.blocks} argument).
  \end{itemize}
\end{itemize}

In practice, the situation is more complicated, as missing data may be
present in blocking/matching variables (such as birth dates) or typos
may occur in names and surnames. Therefore, we developed
\CRANpkg{blocking}, which leverages approximate nearest neighbour (ANN)
algorithms and graphs to create numerous small blocks that can be used
in subsequent analysis. This approach is similar to micro-clustering
(cf.~\citet{johndrow2018theoretical}), but we do not aim at providing the final
linkage of units between or within sources. The basic workflow of the
\CRANpkg{blocking} package consists of the following steps:

\begin{enumerate}
\def\labelenumi{\arabic{enumi}.}
\tightlist
\item
  Create shingles of the input character vectors using
  \CRANpkg{tokenizers} \citep{tokenizers} and \CRANpkg{text2vec}
  \citep{text2vec} packages, or provide a matrix of vectors (e.g.,
  embeddings via \CRANpkg{ragnar}, \citet{ragnar}) representing the input
  character vectors.
\item
  Search for nearest neighbours using ANN algorithms implemented in
  \CRANpkg{rnndescent} \citep{rnndescent}, \CRANpkg{RcppHNSW} \citep{RcppHNSW},
  \CRANpkg{mlpack} (\citet{mlpack2023}, \citet{mlpack2025}), and \CRANpkg{RcppAnnoy}
  \citep{RcppAnnoy}.
\item
  Create final blocks using \CRANpkg{igraph} (\citet{igraph2025},\citet{igraph2006}).
\end{enumerate}

This is the only package in the R ecosystem that readily applies modern
ANN algorithms to reduce the number of comparisons and substantially
accelerate record linkage and deduplication tasks. Additionally, we have
developed the \texttt{pair\_ann()} function for seamless integration with the
\CRANpkg{reclin2} package, as described in the package vignette entitled \href{https://cran.r-project.org/web/packages/blocking/vignettes/v3-integration.html}{\texttt{"Integration\ with\ existing\ packages"}}. However, the \CRANpkg{blocking} package is not limited to the \CRANpkg{reclin2} and can be applied in various record linkage or deduplication contexts.

\subsection{Outline of the article}\label{outline-of-the-article}

This paper is structured as follows. Section~2 provides a description of
the main functionalities of the \CRANpkg{blocking} package and how
results can be evaluated. Section~3 presents two case studies:
probabilistic record linkage and deduplication. These examples
demonstrate how our package can improve the entity resolution pipeline
and integrate with existing R packages.

\section{\texorpdfstring{Blocking of records using \texttt{blocking()} function}{Blocking of records using blocking() function}}\label{sec-blocks}

\subsection{The main function}\label{the-main-function}

The main functionality is available via the \texttt{blocking()} function, which
contains the following key arguments:

\begin{itemize}
\tightlist
\item
  \texttt{x,\ y} -- vectors of keys used for linkage, where \texttt{y\ =\ NULL} indicates a deduplication task;
\item
  \texttt{representation} -- whether \texttt{x} and \texttt{y} should be represented as
  shingles, a custom matrix, or vectors (e.g., provided by the user via the \texttt{model}
  argument);
\item
  \texttt{ann} -- which ANN algorithm should be applied (by default, we use
  the \CRANpkg{rnndescent} package as it supports sparse matrices);
\item
  \texttt{distance} -- which distance metric should be applied (default is
  \texttt{cosine} distance);
\item
  \texttt{graph} -- whether a plot of the graph showing connected records
  should be returned (default \texttt{FALSE});
\item
  \texttt{true\_blocks} -- if a subset of true blocks is available, it can be
  provided here so that quality measures, presented in the next
  section, are returned;
\item
  \texttt{n\_threads} -- number of threads used for computation;
\item
  \texttt{control\_txt} -- controls provided via \texttt{controls\_txt()} specifying
  how \texttt{x,\ y} are processed;
\item
  \texttt{control\_ann} -- controls provided via \texttt{controls\_ann()} allowing
  users to fine-tune the ANN algorithm (see documentation for the
  \texttt{controls\_ann()} function and \texttt{control\_*} functions with names
  referring to specific algorithms, e.g., \texttt{control\_nnd()} for the NND
  algorithm).
\end{itemize}

This function returns an object of class \texttt{blocking} containing the
following elements:

\begin{itemize}
\tightlist
\item
  \texttt{result} -- a \texttt{data.table} with indices (rows) of x, y, block, and
  distance between points;
\item
  \texttt{method} -- name of the ANN algorithm used;
\item
  \texttt{deduplication} -- information about whether deduplication was
  applied;
\item
  \texttt{representation} -- information about whether shingles, a custom matrix, or vectors
  were used;
\item
  \texttt{metrics} -- quality assessment metrics, if \texttt{true\_blocks} is
  provided;
\item
  \texttt{confusion} -- confusion matrix, if \texttt{true\_blocks} is provided;
\item
  \texttt{colnames} -- variable names (\texttt{colnames}) used for search;
\item
  \texttt{graph} -- an \texttt{igraph} class object.
\end{itemize}

\subsection{Assessment of results}\label{sec-assess}

The package implements several measures that can be used to assess
results. The first is the \emph{reduction ratio} (RR), which indicates the
reduction in comparison pairs within the given blocks. It has a value
between \([0,1]\), where 1 indicates perfect reduction whilst values close
to 0 indicate poor reduction. The RR indicator for deduplication has the
following form:

\[
\text{RR}_{\text{dedup}} = 1 - \frac{\sum\limits_{i=1}^{k} \binom{|B_i|}{2}}{\binom{n}{2}},
\]

\noindent where \(k\) is the total number of blocks, \(n\) is the total
number of records in the dataset, and \(|B_i|\) is the number of records
in the \(i\)-th block. \(\sum\limits_{i=1}^{k} \binom{|B_i|}{2}\) is the
number of comparisons after blocking, whilst \(\binom{n}{2}\) is the total
number of possible comparisons without blocking. For record linkage, the
reduction ratio is defined as follows:

\[
\text{RR}_{\text{reclin}} = 1 - \frac{\sum\limits_{i=1}^{k} |B_{i,x}| \cdot |B_{i,y}|} {m \cdot n},
\]

\noindent where \(m\) and \(n\) are the sizes of datasets \(X\) and \(Y\), and
\(k\) is the total number of blocks. The term \(|B_{i,x}|\) is the number of records from dataset \(X\) in the \(i\)-th block, whilst \(|B_{i,y}|\)
is the number of records from dataset \(Y\) in the \(i\)-th block.
The expression \(\sum\limits_{i=1}^{k} |B_{i,x}| \cdot |B_{i,y}|\)
represents the number of comparisons after blocking.

Another way to assess blocking is to examine the confusion matrix at the
\emph{block} level, i.e., blocking results are compared with ground-truth
\emph{blocks} in a pairwise manner (e.g., one true positive pair occurs when
both records from the comparison pair belong to the same predicted
\emph{block} and to the same ground-truth \emph{block} in the evaluation
\texttt{data.frame}). The values in this table are defined as follows:

\begin{itemize}
\tightlist
\item
  True positive (TP): record pairs correctly matched in the same
  block.
\item
  False positive (FP): record pairs identified as matches that are not
  true matches in the same block.
\item
  True negative (TN): record pairs correctly identified as non-matches
  (different blocks).
\item
  False negative (FN): record pairs identified as non-matches that are
  true matches in the same block.
\end{itemize}

Metrics calculated based on this confusion matrix are presented in
Table~1.

\begin{longtable}[]{@{}
  >{\raggedright\arraybackslash}p{(\linewidth - 6\tabcolsep) * \real{0.2500}}
  >{\raggedright\arraybackslash}p{(\linewidth - 6\tabcolsep) * \real{0.2500}}
  >{\raggedright\arraybackslash}p{(\linewidth - 6\tabcolsep) * \real{0.2500}}
  >{\raggedright\arraybackslash}p{(\linewidth - 6\tabcolsep) * \real{0.2500}}@{}}
\caption{Evaluation Metrics}\tabularnewline
\toprule\noalign{}
\begin{minipage}[b]{\linewidth}\raggedright
\textbf{Metric}
\end{minipage} & \begin{minipage}[b]{\linewidth}\raggedright
\textbf{Formula}
\end{minipage} & \begin{minipage}[b]{\linewidth}\raggedright
\textbf{Metric}
\end{minipage} & \begin{minipage}[b]{\linewidth}\raggedright
\textbf{Formula}
\end{minipage} \\
\midrule\noalign{}
\endfirsthead
\toprule\noalign{}
\begin{minipage}[b]{\linewidth}\raggedright
\textbf{Metric}
\end{minipage} & \begin{minipage}[b]{\linewidth}\raggedright
\textbf{Formula}
\end{minipage} & \begin{minipage}[b]{\linewidth}\raggedright
\textbf{Metric}
\end{minipage} & \begin{minipage}[b]{\linewidth}\raggedright
\textbf{Formula}
\end{minipage} \\
\midrule\noalign{}
\endhead
\bottomrule\noalign{}
\endlastfoot
Recall & \(\frac{TP}{TP + FN}\) & Accuracy & \(\frac{TP + TN}{TP + TN + FP + FN}\) \\
Precision & \(\frac{TP}{TP + FP}\) & Specificity & \(\frac{TN}{TN + FP}\) \\
F1 Score & \(2 \cdot \frac{\text{Precision} \times \text{Recall}}{\text{Precision} + \text{Recall}}\) & False Positive Rate & \(\frac{FP}{FP + TN}\) \\
False Negative Rate & \(\frac{FN}{FN + TP}\) & & \\
\end{longtable}

\section{Case studies}\label{sec-case}

\subsection{An example of blocking for record linkage}\label{an-example-of-blocking-for-record-linkage}

Let us first load the required packages.

\begin{verbatim}
library("blocking")
library("data.table")
library("reclin2")
\end{verbatim}

We demonstrate the use of the \texttt{blocking()} function for record linkage
on the \texttt{foreigners} dataset included in the package. This fictional
representation of the foreign population in Poland was generated based
on publicly available information, preserving the distributions from
administrative registers. It contains 110,000 rows with 100,000 entities
(thus containing 10,000 duplicates). Each row represents one record,
with the following columns: \texttt{fname} -- first name, \texttt{sname} -- second
name, \texttt{surname} -- surname, \texttt{date} -- date of birth, \texttt{region} -- region
(county), \texttt{country} -- country, and \texttt{true\_id} -- a person identifier.

Next, we load the data and examine the first six records.

\begin{verbatim}
data("foreigners")
setDT(foreigners)
head(foreigners)
\end{verbatim}

\begin{verbatim}
#>     fname  sname    surname       date region country true_id
#>    <char> <char>     <char>     <char> <char>  <char>   <num>
#> 1:   emin            imanov 1998/02/05            031       0
#> 2: nurlan        suleymanli 2000/08/01            031       1
#> 3:   amio        maharrsmov 1939/03/08            031       2
#> 4:   amik        maharramof 1939/03/08            031       2
#> 5:   amil        maharramov 1993/03/08            031       2
#> 6:  gadir        jahangirov 1991/08/29            031       3
\end{verbatim}

In the next step, we split the dataset into two separate \texttt{data.frame}s:
one containing the first appearance of each entity in the \texttt{foreigners}
dataset, and the other containing subsequent appearances. We then add
row identifiers (\texttt{x} and \texttt{y}).

\begin{verbatim}
foreigners_1 <- foreigners[!duplicated(foreigners$true_id), ]
foreigners_1[, x := 1:.N]
foreigners_2 <- foreigners[duplicated(foreigners$true_id), ]
foreigners_2[, y := 1:.N]
\end{verbatim}

Now, in both datasets we remove separators from the date column and
create a new character column that concatenates information from all
columns (excluding \texttt{true\_id}) in each row. Information stored in the
\texttt{txt} column will be used for blocking records in the \texttt{blocking()}
function.

\begin{verbatim}
foreigners_1[, txt := paste0(fname, sname, surname, gsub("/", "", date), 
                             region, country)]
foreigners_2[, txt := paste0(fname, sname, surname, gsub("/", "", date), 
                             region, country)]
head(foreigners_1[, .(true_id, txt)])
\end{verbatim}

\begin{verbatim}
#>    true_id                           txt
#>      <num>                        <char>
#> 1:       0         eminimanov19980205031
#> 2:       1   nurlansuleymanli20000801031
#> 3:       2     amiomaharrsmov19390308031
#> 4:       3    gadirjahangirov19910829031
#> 5:       4 zaurbayramova1996100601261031
#> 6:       5       asifmammadov19970726031
\end{verbatim}

The default algorithm is the Nearest Neighbour Descent Method
\citep{Dong2011} implemented in the \CRANpkg{rnndescent} package. Note that a
default parameter of the \texttt{blocking()} function is \texttt{seed\ =\ 2023}, which sets
the random seed.

\begin{verbatim}
result_reclin <- blocking(x = foreigners_1$txt,
                          y = foreigners_2$txt)
\end{verbatim}

Now, we can examine the results by printing the \texttt{result\_reclin} object.
In this example, we have created
6,470
blocks utilizing 1,232
2-character shingles. Blocks are small, as we have
3,920 blocks of 2
elements, 1,599 blocks of
3 elements, \ldots,
2 blocks of
7 elements.

\begin{verbatim}
result_reclin
\end{verbatim}

\begin{verbatim}
#> ========================================================
#> Blocking based on the nnd method.
#> Number of blocks: 6470.
#> Number of columns used for blocking: 1232.
#> Reduction ratio: 0.9999.
#> ========================================================
#> Distribution of the size of the blocks:
#>    2    3    4    5    6    7 
#> 3920 1599  928   19    2    2
\end{verbatim}

To access the result, one should use \texttt{result\_reclin\$result}. The
resulting data.table has four columns (as presented below):

\begin{itemize}
\tightlist
\item
  \texttt{x} -- reference dataset (i.e., \texttt{foreigners\_1}) -- this may not
  contain all units of \texttt{foreigners\_1};
\item
  \texttt{y} -- query (each row of \texttt{foreigners\_2}) -- this will contain all
  units of \texttt{foreigners\_2};
\item
  \texttt{block} -- the block identifier;
\item
  \texttt{dist} -- distance between pairs.
\end{itemize}

\begin{verbatim}
head(result_reclin$result)
\end{verbatim}

\begin{verbatim}
#>        x     y block      dist
#>    <int> <int> <num>     <num>
#> 1:     3     1     1 0.2216882
#> 2:     3     2     1 0.2122737
#> 3:    21     3     2 0.1172652
#> 4:    57     4     3 0.1863238
#> 5:    57     5     3 0.1379310
#> 6:    61     6     4 0.2307692
\end{verbatim}

Let's examine the first block. Clearly, there are typos in the \texttt{fname}
and \texttt{surname}. Nevertheless, all records refer to the same entity (as
denoted by \texttt{true\_id}).

\begin{verbatim}
rbind(foreigners_1[3, 1:7], foreigners_2[1:2, 1:7])
\end{verbatim}

\begin{verbatim}
#>     fname  sname    surname       date region country true_id
#>    <char> <char>     <char>     <char> <char>  <char>   <num>
#> 1:   amio        maharrsmov 1939/03/08            031       2
#> 2:   amik        maharramof 1939/03/08            031       2
#> 3:   amil        maharramov 1993/03/08            031       2
\end{verbatim}

Now we use the \texttt{true\_id} column to evaluate our approach.

\begin{verbatim}
matches <- merge(x = foreigners_1[, .(x, true_id)],
                 y = foreigners_2[, .(y, true_id)],
                 by = "true_id")
matches[, block := rleid(x)]
head(matches)
\end{verbatim}

\begin{verbatim}
#> Key: <true_id>
#>    true_id     x     y block
#>      <num> <int> <int> <int>
#> 1:       2     3     1     1
#> 2:       2     3     2     1
#> 3:      20    21     3     2
#> 4:      56    57     4     3
#> 5:      56    57     5     3
#> 6:      60    61     6     4
\end{verbatim}

We have 10,000 matched pairs, which can be used in the \texttt{true\_blocks}
argument of the \texttt{blocking()} function to specify the true block
assignments. This \texttt{data.frame} must contain three columns:
\texttt{x}, \texttt{y}, and \texttt{block}.

We obtain quality metrics for the assessment of record
linkage.

\begin{verbatim}
res_reclin <- blocking(x = foreigners_1$txt,
                       y = foreigners_2$txt,
                       true_blocks = matches[, .(x, y, block)])
\end{verbatim}

\begin{verbatim}
res_reclin
\end{verbatim}

\begin{verbatim}
#> ========================================================
#> Blocking based on the nnd method.
#> Number of blocks: 6470.
#> Number of columns used for blocking: 1232.
#> Reduction ratio: 0.9999.
#> ========================================================
#> Distribution of the size of the blocks:
#>    2    3    4    5    6    7 
#> 3920 1599  928   19    2    2 
#> ========================================================
#> Evaluation metrics (standard, in %):
#>      recall   precision         fpr         fnr    accuracy specificity 
#>     96.7532     78.6700      0.0038      3.2468     99.9957     99.9962 
#>    f1_score 
#>     86.7795
\end{verbatim}

For example, our approach results in a~3.25\% FNR. To improve this,
we can increase the epsilon parameter of the NND method from 0.1 to 0.5.
To do so, we configure the \texttt{control\_ann} parameter in the \texttt{blocking()}
function using the \texttt{controls\_ann()} and \texttt{control\_nnd()} functions.

\begin{verbatim}
res_reclin2 <- blocking(x = foreigners_1$txt,
                        y = foreigners_2$txt,
                        true_blocks = matches[, .(x, y, block)],
                        control_ann = controls_ann(nnd = control_nnd(epsilon = 0.5)))
\end{verbatim}

\begin{verbatim}
res_reclin2
\end{verbatim}

\begin{verbatim}
#> ========================================================
#> Blocking based on the nnd method.
#> Number of blocks: 6394.
#> Number of columns used for blocking: 1232.
#> Reduction ratio: 0.9999.
#> ========================================================
#> Distribution of the size of the blocks:
#>    2    3    4    5    7 
#> 3800 1615  954   21    4 
#> ========================================================
#> Evaluation metrics (standard, in %):
#>      recall   precision         fpr         fnr    accuracy specificity 
#>     96.8776     80.0500      0.0036      3.1224     99.9960     99.9964 
#>    f1_score 
#>     87.6636
\end{verbatim}

That decreases the FNR to
3.12\%.

Now, to use the result in the record linkage process, we add this
information to both datasets and specify it in the appropriate argument
of a given function. Below, we present an example using the \CRANpkg{reclin2}
package, employing the Expectation-Maximization algorithm by \citet{fellegi1969theory}.

\begin{verbatim}
foreigners_1[res_reclin2$result, on = "x", block:= i.block]
foreigners_2[res_reclin2$result, on = "y", block:= i.block]

pairs <- pair_blocking(x = foreigners_1, 
              y = foreigners_2, on = "block") |>
          compare_pairs(c("fname", "surname", "date")) 
model <- problink_em(~ fname + surname + date, data = pairs)
predict(model, pairs = pairs, add = TRUE, type = "all") |>
  head(n = 4)
\end{verbatim}

\begin{verbatim}
#>   First data set:  100 000 records
#>   Second data set: 10 000 records
#>   Total number of pairs: 4 pairs
#>   Blocking on: 'block'
#> 
#>       .x    .y  fname surname   date      mprob    uprob      mpost     upost
#>    <int> <int> <lgcl>  <lgcl> <lgcl>      <num>    <num>      <num>     <num>
#> 1:     3     1  FALSE   FALSE   TRUE 0.24865523 0.217235 0.12223165 0.8777683
#> 2:     3     2  FALSE   FALSE  FALSE 0.09215071 0.563555 0.01950491 0.9804951
#> 3:    21     3  FALSE   FALSE   TRUE 0.24865523 0.217235 0.12223165 0.8777683
#> 4:    57     4  FALSE   FALSE  FALSE 0.09215071 0.563555 0.01950491 0.9804951
#>        weight
#>         <num>
#> 1:  0.1350875
#> 2: -1.8108395
#> 3:  0.1350875
#> 4: -1.8108395
\end{verbatim}

\subsection{An example of blocking for deduplication}\label{an-example-of-blocking-for-deduplication}

In this section, we demonstrate a~deduplication application using the
\texttt{blocking()} function on the \texttt{RLdata500} dataset from the
\CRANpkg{RecordLinkage} package. Note that the dataset is included in
the \CRANpkg{blocking} package. It contains artificial personal data, and fifty
records have been duplicated with randomly generated errors. Each row
represents one record, with the following columns: \texttt{fname\_c1} -- first
name, \texttt{fname\_c2} -- second name, \texttt{lname\_c1} -- last name, \texttt{lname\_c2} --
last name (second component), \texttt{by}, \texttt{bm}, \texttt{bd} -- year, month, and day
of birth, \texttt{rec\_id} -- record ID, and \texttt{ent\_id} -- entity ID.

\begin{verbatim}
data("RLdata500")
setDT(RLdata500)
head(RLdata500)
\end{verbatim}

\begin{verbatim}
#>    fname_c1 fname_c2 lname_c1 lname_c2    by    bm    bd rec_id ent_id
#>      <char>   <char>   <char>   <char> <int> <int> <int>  <int>  <int>
#> 1:  CARSTEN             MEIER           1949     7    22      1     34
#> 2:     GERD             BAUER           1968     7    27      2     51
#> 3:   ROBERT          HARTMANN           1930     4    30      3    115
#> 4:   STEFAN             WOLFF           1957     9     2      4    189
#> 5:     RALF           KRUEGER           1966     1    13      5     72
#> 6:  JUERGEN            FRANKE           1929     7     4      6    142
\end{verbatim}

For the purpose of this example, we create a new column (\texttt{id\_count})
that indicates how many times a given unit occurs, and then add leading
zeros to the \texttt{bm} and \texttt{bd} columns. Finally, we create a new string
column that concatenates information from all columns (excluding
\texttt{rec\_id}, \texttt{ent\_id}, and \texttt{id\_count}), as presented below.

\begin{verbatim}
RLdata500[, id_count :=.N, ent_id]
RLdata500[, txt:=tolower(paste0(fname_c1,fname_c2,lname_c1,lname_c2,by,
                                sprintf("%02d", bm),sprintf("%02d", bd)))]
head(RLdata500[, .(rec_id, id_count, txt)])
\end{verbatim}

\begin{verbatim}
#>    rec_id id_count                    txt
#>     <int>    <int>                 <char>
#> 1:      1        1   carstenmeier19490722
#> 2:      2        2      gerdbauer19680727
#> 3:      3        1 roberthartmann19300430
#> 4:      4        1    stefanwolff19570902
#> 5:      5        1    ralfkrueger19660113
#> 6:      6        1  juergenfranke19290704
\end{verbatim}

As in the previous example, we use the \texttt{txt} column in the \texttt{blocking()}
function. This time, we set \texttt{ann\ =\ "hnsw"} to use the Hierarchical
Navigable Small World (HNSW; \citet{malkov2018efficient}) algorithm from the
\CRANpkg{RcppHNSW} package.

\begin{verbatim}
res_dedup <- blocking(x = RLdata500$txt,
                      ann = "hnsw",
                      verbose = 1)
\end{verbatim}

\begin{verbatim}
#> ===== creating tokens =====
#> ===== starting search (hnsw, x, y: 500, 500, t: 429) =====
#> ===== creating graph =====
\end{verbatim}

The results are as follows. This time, the HNSW algorithm provided
blocks varying from 2 to 17 units.

\begin{verbatim}
res_dedup
\end{verbatim}

\begin{verbatim}
#> ========================================================
#> Blocking based on the hnsw method.
#> Number of blocks: 133.
#> Number of columns used for blocking: 429.
#> Reduction ratio: 0.9916.
#> ========================================================
#> Distribution of the size of the blocks:
#>  2  3  4  5  6  7  8  9 10 11 12 17 
#> 46 35 23  8  6  6  2  3  1  1  1  1
\end{verbatim}

Next, we create a long \texttt{data.table} with information on blocks and units
from the original dataset. We add the block information to the final
dataset. We can check in how many blocks the same entities (\texttt{ent\_id})
are observed. In our example, all identical entities are in the same
blocks.

\begin{verbatim}
# Reshape to long format
df_block_melted <- melt(res_dedup$result, id.vars = c("block", "dist"))
# Unique record-block pairs
df_block_melted_rec_block <- unique(df_block_melted[, .(rec_id=value, block)])
# Join block IDs
RLdata500[df_block_melted_rec_block, on = "rec_id", block_id := i.block]
# Check block consistency per entity
RLdata500[, .(uniq_blocks = uniqueN(block_id)), .(ent_id)][, .N, uniq_blocks]
\end{verbatim}

\begin{verbatim}
#>    uniq_blocks     N
#>          <int> <int>
#> 1:           1   450
\end{verbatim}

Additionally, Figure \ref{fig:density-plot} visualises the results based on whether a block contains
matches or not.

\begin{figure}[!ht]
\includegraphics[width=0.8\linewidth,alt={A density plot of distances between units that are true matches (red) and non-matches (blue) within blocks created by the \CRANpkg{blocking}. The distribution of distance for matches is bimodal. There is a group of units that are true matches where the distance between them is small (less than 0.2), whilst for the second group, the distance is similar to true non-matches (between 0.4 and 0.6). This distance may be used as additional information for deduplication (and record linkage) studies.}]{./figures/fig-1-density} \caption{Distribution of distances between true matches and non-matches within blocks}\label{fig:density-plot}
\end{figure}

Finally, we compare the evaluation metrics across all ANN algorithms
supported by the \texttt{blocking()} function, i.e., NND, HNSW, Annoy (from the
\CRANpkg{RcppAnnoy} package), Locality-Sensitive Hashing (LSH, from the
\CRANpkg{mlpack} package), and \(k\)-Nearest Neighbours (kNN -- denoted as
\texttt{"kd"}, from the \CRANpkg{mlpack} package). We use the \texttt{rec\_id} and
\texttt{ent\_id} columns from the \texttt{RLdata500} dataset to specify the true blocks
and then calculate evaluation metrics for all algorithms.

We compare our package with the \texttt{klsh()} function from the
\CRANpkg{klsh} package, configured to create 10 blocks (denoted as
\texttt{klsh\_10}) and 100 blocks (denoted as \texttt{klsh\_100}), respectively. In both
settings, we use 20 random projections and 2-character shingles. The
results are presented in Table 2, and the code generating it
is included in the Supplementary materials.

\begin{table}

\caption{\label{tab:table-latex}Comparison of various approximate nearest neighbour algorithms implemented in the   \CRANpkg{blocking} and the \CRANpkg{klsh} package for creation of blocks for deduplication (values in percentages)}
\centering
\begin{tabular}[t]{l|r|r|r|r|r|r|r}
\hline
  & recall & precision & fpr & fnr & accuracy & specificity & f1\_score\\
\hline
nnd & 100 & 5.17 & 0.74 & 0 & 99.26 & 99.26 & 9.83\\
\hline
hnsw & 100 & 4.76 & 0.80 & 0 & 99.20 & 99.20 & 9.08\\
\hline
annoy & 100 & 4.80 & 0.79 & 0 & 99.21 & 99.21 & 9.17\\
\hline
lsh & 98 & 1.04 & 3.74 & 2 & 96.26 & 96.26 & 2.06\\
\hline
kd & 100 & 5.19 & 0.73 & 0 & 99.27 & 99.27 & 9.87\\
\hline
klsh\_10 & 84 & 0.33 & 10.13 & 16 & 89.87 & 89.87 & 0.66\\
\hline
klsh\_100 & 90 & 3.72 & 0.94 & 10 & 99.06 & 99.06 & 7.14\\
\hline
\end{tabular}
\end{table}

The results demonstrate a clear performance hierarchy among the ANN
algorithms implemented in the \CRANpkg{blocking} package, with traditional
tree-based methods (NND, HNSW, Annoy, and kNN) achieving perfect recall
(100\%) whilst maintaining excellent precision and F1 scores around
5-10\%. Notably, these methods exhibit minimal FPR (0.73-0.80\%) and
maintain high specificity (99.20-99.27\%), indicating their effectiveness
in creating tight, accurate blocks.

In contrast, the LSH-based methods show more variable performance: the
\texttt{mlpack} LSH implementation achieves 98\% recall but suffers from higher
FPR (3.74\%) and, importantly, FNR (2\%), whilst the \texttt{klsh} package
results reveal a trade-off between block granularity and performance --
\texttt{klsh\_10} with only 10 blocks shows poor recall (84\%), high FPR
(10.13\%), and FNR (16\%), whereas \texttt{klsh\_100} with 100 blocks recovers
much of the performance (90\% recall, 0.94\% FPR, but high FNR of 10\%).
This indicates that the \texttt{klsh}, particularly these implementations of
the LSH approach, misses a large number of true matches.

These findings suggest that modern ANN algorithms like NND, HNSW, and
Annoy provide superior blocking performance for entity resolution tasks,
offering both computational efficiency and high-quality results that
minimise both missed matches and false linkages.

\section{Summary}\label{summary}

In this paper, we have demonstrated the basic use cases of the
\CRANpkg{blocking} package. We believe that the software will be useful
for researchers working in various fields where integration of multiple
sources is an important aspect. This is certainly of interest in the
field of official statistics, where register-based statistics rely on
high-quality linkage of administrative datasets, or medical studies,
where assessment of health statistics relies on correct linkage of
medical history with treatment outcomes or mortality records.

Furthermore, for users interested in integration with the
\CRANpkg{reclin2} package, we refer to the documentation of the
\texttt{pair\_ann()} function and the vignette entitled
\href{https://cran.r-project.org/web/packages/blocking/vignettes/v3-integration.html}{\texttt{"Integration\ with\ existing\ packages"}},
which provides case studies demonstrating how the \CRANpkg{blocking}
package can be included in existing record linkage/deduplication
pipelines.

\section{Acknowledgements}\label{acknowledgements}

Work on this package is supported by the National Science Centre, OPUS
20 grant no. 2020/39/B/HS4/00941. We also thank participants of the uRos
2024 conference for valuable comments and discussion.

We have also developed a Python version of the package,
\href{https://blockingpy.readthedocs.io/en/latest/}{\texttt{BlockingPy}}, which is
available through PyPI. It has a similar structure but offers more ANN
algorithms (e.g., FAISS) and enables the use of embeddings. For more
details, see: Strojny, T., \& Beręsewicz, M. (2025). BlockingPy:
Approximate nearest neighbours for blocking of records for entity
resolution. arXiv preprint arXiv:2504.04266.

\bibliography{RJreferences.bib}

\address{%
Maciej Beręsewicz\\
University of Economics and BusinessStatisical Office in Poznań\\%
Department of Statistics\\ Centre for the Methodology of Population Studies\\
%
\url{https://maciejberesewicz.com}\\%
\textit{ORCiD: \href{https://orcid.org/0000-0002-8281-4301}{0000-0002-8281-4301}}\\%
\href{mailto:maciej.beresewicz@poznan.pl}{\nolinkurl{maciej.beresewicz@poznan.pl}}%
}

\address{%
Adam Struzik\\
Adam Mickiewicz UniversityStatisical Office in Poznań\\%
Faculty of Mathematics and Computer Science\\ Centre for Urban Statistics\\
%
%
%
\href{mailto:adastr5@st.amu.edu.pl}{\nolinkurl{adastr5@st.amu.edu.pl}}%
}
